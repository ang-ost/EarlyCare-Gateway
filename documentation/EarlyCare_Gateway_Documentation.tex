\documentclass[12pt,a4paper]{report}
\usepackage[utf8]{inputenc}
\usepackage[T1]{fontenc}
\usepackage[english]{babel}
\usepackage{geometry}
\usepackage{graphicx}
\usepackage{hyperref}
\usepackage{listings}
\usepackage{xcolor}
\usepackage{fancyhdr}
\usepackage{titlesec}
\usepackage{tocloft}
\usepackage{booktabs}
\usepackage{longtable}
\usepackage{enumitem}
\usepackage{amsmath}
\usepackage{tikz}
\usepackage{float}
\usetikzlibrary{shapes.geometric, arrows, positioning}

% Page geometry
\geometry{
    left=2.5cm,
    right=2.5cm,
    top=3cm,
    bottom=3cm
}

% Colors
\definecolor{primaryblue}{RGB}{41, 98, 255}
\definecolor{codegray}{RGB}{245, 245, 245}
\definecolor{codegreen}{RGB}{0, 128, 0}
\definecolor{codepurple}{RGB}{128, 0, 128}

% Hyperref setup
\hypersetup{
    colorlinks=true,
    linkcolor=primaryblue,
    filecolor=primaryblue,
    urlcolor=primaryblue,
    citecolor=primaryblue,
    pdftitle={MAV - Technical Documentation},
    pdfauthor={Development Team}
}

% Code listing style
\lstdefinestyle{codestyle}{
    backgroundcolor=\color{codegray},
    commentstyle=\color{codegreen},
    keywordstyle=\color{primaryblue}\bfseries,
    stringstyle=\color{codepurple},
    basicstyle=\ttfamily\footnotesize,
    breakatwhitespace=false,
    breaklines=true,
    captionpos=b,
    keepspaces=true,
    numbers=left,
    numbersep=5pt,
    numberstyle=\tiny\color{gray},
    showspaces=false,
    showstringspaces=false,
    showtabs=false,
    tabsize=2,
    frame=single,
    rulecolor=\color{gray!30}
}
\lstset{style=codestyle}

% Header and footer
\pagestyle{fancy}
\fancyhf{}
\fancyhead[L]{\leftmark}
\fancyhead[R]{MAV}
\fancyfoot[C]{\thepage}
\renewcommand{\headrulewidth}{0.4pt}
\renewcommand{\footrulewidth}{0.4pt}

% Chapter and section formatting
\titleformat{\chapter}[display]
{\normalfont\huge\bfseries\color{primaryblue}}{\chaptertitlename\ \thechapter}{20pt}{\Huge}
\titleformat{\section}
{\normalfont\Large\bfseries\color{primaryblue}}{\thesection}{1em}{}
\titleformat{\subsection}
{\normalfont\large\bfseries}{\thesubsection}{1em}{}

\begin{document}

% Title Page
\begin{titlepage}
    \centering
    \vspace*{2cm}
    
    {\Huge\bfseries\color{primaryblue} MAV - Medical Access \& Vision\par}
    \vspace{0.5cm}
    {\Large\itshape Intelligent Clinical Decision Support System\par}
    
    \vspace{2cm}
    
    \includegraphics[width=0.5\textwidth]{img/logo.png}\par
    
    \vspace{2cm}
    
    {\large\bfseries Technical Documentation\par}
    \vspace{0.5cm}
    {\large Version 1.0.0\par}
    
    \vfill
    
    \begin{tabular}{ll}
        \textbf{Project Type:} & Clinical Decision Support System \\
        \textbf{Platform:} & Web Application \\
        \textbf{Backend:} & Python Flask \\
        \textbf{Frontend:} & React 18 (Vite) \\
        \textbf{Database:} & MongoDB \\
        \textbf{AI Engine:} & Google Gemini \\
    \end{tabular}
    
    \vspace{1.5cm}
    
    {\large January 2026\par}
\end{titlepage}

% Table of Contents
\tableofcontents
\newpage

% Abstract
\chapter*{Abstract}
\addcontentsline{toc}{chapter}{Abstract}

\textbf{MAV} is an advanced, production-ready software solution designed to bridge the gap between raw clinical data and modern Artificial Intelligence. This comprehensive system serves as an intelligent routing and processing engine that ingests multi-modal data (text, vital signs, and medical images), validates the information, enriches it with calculated metadata, and leverages state-of-the-art AI to assist healthcare professionals in early disease diagnosis.

The platform addresses critical challenges in modern healthcare environments: data fragmentation through unified data source management, latency through real-time decision support, privacy through strict HIPAA/GDPR compliance mechanisms, and reliability through graceful failure handling with fallback mechanisms.

This documentation provides a complete technical overview of the system architecture, implementation details, deployment procedures, and future improvement opportunities for the MAV platform.

\newpage

%==========================================
% CHAPTER 1: INTRODUCTION
%==========================================
\chapter{Introduction}

\section{Overview and Vision}

The healthcare industry faces unprecedented challenges in managing the ever-increasing volume of clinical data while maintaining high standards of patient care. MAV emerges as a sophisticated solution to address these multifaceted challenges, providing healthcare professionals with an intelligent platform that not only manages patient records but also offers AI-powered diagnostic assistance.

\subsection{Problem Statement}

Modern healthcare systems struggle with several critical issues:

\begin{enumerate}[label=\arabic*.]
    \item \textbf{Data Fragmentation}: Patient information is often scattered across multiple systems, making it difficult to obtain a comprehensive view of a patient's medical history.
    
    \item \textbf{Decision Latency}: Healthcare professionals need rapid access to relevant information and analytical insights, especially in emergency triage situations.
    
    \item \textbf{Privacy Concerns}: The integration of cloud-based AI services requires stringent data protection mechanisms to comply with regulations such as HIPAA and GDPR.
    
    \item \textbf{System Reliability}: Medical systems must handle failures gracefully without compromising patient care continuity.
\end{enumerate}
\pagebreak
\subsection{Solution Approach}

MAV addresses these challenges through a carefully designed architecture that implements:

\begin{itemize}
    \item A unified gateway pattern for clinical data processing
    \item Real-time AI-powered diagnostic assistance using Google Gemini
    \item Robust authentication and authorization mechanisms
    \item Comprehensive audit logging for accountability
    \item Flexible deployment options via Docker containerization
\end{itemize}

\section{Key Features}

The platform offers a comprehensive feature set designed to enhance clinical workflows:

\begin{itemize}
    \item \textbf{Patient Management}: Complete patient lifecycle management with support for both Italian citizens (using Fiscal Code validation) and foreign patients (automatic ID generation).
    
    \item \textbf{Clinical Record Management}: Comprehensive clinical record creation, editing, and retrieval with support for vital signs, symptoms, notes, and file attachments.
    
    \item \textbf{AI-Powered Diagnostics}: Integration with Google Gemini for multimodal analysis of clinical data and medical images.
    
    \item \textbf{Medical Chatbot}: An AI-powered conversational assistant for medical queries.
    
    \item \textbf{Professional Reporting}: PDF export capabilities for clinical documentation.
    
    \item \textbf{Doctor Authentication}: Secure registration and login system with unique mnemonic doctor IDs.
\end{itemize}

\newpage

%==========================================
% CHAPTER 2: TECHNICAL ARCHITECTURE
%==========================================
\chapter{Technical Architecture}

\section{System Overview}

MAV follows a \textbf{Layered Architecture} with a central Gateway component orchestrating the entire data flow pipeline. This architectural approach ensures separation of concerns, maintainability, and scalability.

\subsection{The Gateway Concept}

The core of the backend is the \texttt{ClinicalGateway}, which differs fundamentally from a standard CRUD controller. The Gateway treats every incoming clinical request as a payload that must pass through a strict pipeline of handlers, ensuring that no data is ever processed without proper validation, anonymization, and auditing.

\subsection{Architecture Diagram}

\begin{figure}[H]
\centering
\includegraphics[width=0.5\textwidth]{img/architecture-diagram.png}
\caption{MAV System Architecture}
\end{figure}

\section{Data Flow Pipeline}

The clinical data processing follows a well-defined pipeline:

\begin{enumerate}[label=\textbf{Step \arabic*:}]
    \item \textbf{Ingestion}: The React frontend sends a Patient Record containing symptoms, vital signs, and attachments via HTTP POST request.
    
    \item \textbf{Gateway Entry}: The request enters the \texttt{ClinicalGateway} through the Flask API layer, where initial request parsing occurs.
    
    \item \textbf{Processing Chain}: The request passes through a chain of specialized handlers:
    \begin{itemize}
        \item \texttt{ValidationHandler}: Verifies data integrity, including valid Fiscal Codes and realistic vital sign ranges.
        \item \texttt{EnrichmentHandler}: Adds calculated metadata such as patient age from date of birth and BMI calculations.
        \item \texttt{PrivacyHandler}: Anonymizes sensitive fields before external AI processing.
        \item \texttt{TriageHandler}: Calculates initial urgency scores based on configured clinical rules.
    \end{itemize}
    
    \item \textbf{Strategy Execution}: The system selects the appropriate AI strategy (e.g., \texttt{GeminiStrategy}) to analyze the clinical data.
    
    \item \textbf{Observer Notification}: Auditing and Metrics systems record the transaction results for compliance and analytics.
    
    \item \textbf{Persistence}: Validated and enriched data is stored in MongoDB collections.
    
    \item \textbf{Response}: The complete, analyzed result is returned to the Frontend for display.
\end{enumerate}

\section{Component Interaction}

\subsection{Frontend-Backend Communication}

The frontend communicates with the backend through a RESTful API interface. All requests include credentials for session management:

\begin{lstlisting} 
const response = await fetch(getApiUrl('/api/patient/search'), {
    method: 'POST',
    headers: { 'Content-Type': 'application/json' },
    credentials: 'include',
    body: JSON.stringify({ fiscal_code: searchCode })
});
\end{lstlisting}

\subsection{CORS Configuration}

Cross-Origin Resource Sharing is configured to allow the frontend application to communicate with the backend API:

\begin{lstlisting}[language=Python, caption=CORS Configuration]
CORS(app, 
     resources={
         r"/api/*": {
             "origins": allowed_origins,
             "methods": ["GET", "POST", "PUT", "DELETE", "OPTIONS"],
             "allow_headers": ["Content-Type", "Authorization"],
             "supports_credentials": True
         }
     })
\end{lstlisting}

\newpage

%==========================================
% CHAPTER 3: DESIGN PATTERNS
%==========================================
\chapter{Design Patterns Analysis}

The system implements several GoF (Gang of Four) design patterns to ensure maintainability and extensibility.

\section{Implemented Patterns}

\subsection{Strategy Pattern (Partially Implemented)}
The AI module is designed to support multiple strategies, currently using Google Gemini as the sole provider. The architecture easily allows adding other AI providers (OpenAI, Claude, etc.) through common interfaces.

\subsection{Decorator Pattern (Implemented)}
Used for authentication via the \texttt{@require\_login} decorator that protects sensitive routes by verifying user session presence.

\begin{lstlisting}[language=Python, caption=Authentication Decorator]
def require_login(f):
    @wraps(f)
    def decorated_function(*args, **kwargs):
        if 'doctor_id' not in session:
            return jsonify({'error': 'Unauthorized'}), 401
        return f(*args, **kwargs)
    return decorated_function
\end{lstlisting}
\pagebreak
\section{Conceptual Patterns}

\subsection{Chain of Responsibility}
Described in the architecture for data processing pipeline (validation, enrichment, privacy, triage). Currently implemented inline in routes, but ready for formal refactoring.

\subsection{Observer Pattern}
Planned for audit logging and metrics. Currently the system uses direct logging, but the architecture supports adding observers for system events.

\subsection{Facade Pattern}
Planned for future integrations with external systems (HL7/FHIR, PACS, LIS).

\section{Summary}

\begin{table}[h]
\centering
\begin{tabular}{llp{6cm}}
\toprule
\textbf{Pattern} & \textbf{Status} & \textbf{Notes} \\
\midrule
Strategy & Partial & AI module ready \\
Decorator & Implemented & Route authentication \\
Chain of Responsibility & Conceptual & Inline implementation \\
Observer & Conceptual & Direct logging \\
Facade & Planned & Future integrations \\
\bottomrule
\end{tabular}
\caption{Pattern Implementation Status}
\end{table}

\newpage

%==========================================
% CHAPTER 4: TECHNOLOGY STACK
%==========================================
\chapter{Technology Stack}

MAVutilizes a modern, well-integrated technology stack designed for reliability, scalability, and maintainability.

\section{Backend Technologies}

\subsection{Python Flask Framework}
The backend is built using Flask, a lightweight micro-framework that provides flexibility without imposing unnecessary constraints.

\begin{table}[h]
\centering
\begin{tabular}{ll}
\toprule
\textbf{Component} & \textbf{Version/Details} \\
\midrule
Python & 3.8+ \\
Flask & 3.0.0+ \\
Flask-CORS & 4.0.0+ \\
\bottomrule
\end{tabular}
\caption{Core Backend Framework}
\end{table}

\subsection{AI Integration}
\begin{itemize}
    \item \textbf{Google Generative AI SDK}: \texttt{google-generativeai >= 0.3.0}
    \item Model: \texttt{gemini-3-flash-preview}
    \item Supports multimodal input (text + images)
\end{itemize}

\subsection{Data Processing Libraries}
\begin{itemize}
    \item \textbf{NumPy}: Numerical computing
    \item \textbf{Pandas}: Data manipulation and analysis
    \item \textbf{PIL (Pillow)}: Image processing for medical attachments
    \item \textbf{PyPDF2}: PDF parsing and generation
    \item \textbf{ReportLab}: Professional PDF report generation
\end{itemize}

\subsection{Database Connectivity}
\begin{itemize}
    \item \textbf{PyMongo}: MongoDB driver for Python (>= 4.6.0)
    \item SSL/TLS support for secure connections
    \item Connection pooling for performance
\end{itemize}

\section{Frontend Technologies}

\subsection{React 18 with Vite}
The frontend leverages modern React development practices:

\begin{table}[h]
\centering
\begin{tabular}{ll}
\toprule
\textbf{Technology} & \textbf{Version/Details} \\
\midrule
React & 18.2.0+ \\
React DOM & 18.2.0+ \\
Vite & 5.0.8+ \\
@vitejs/plugin-react & 4.2.1+ \\
\bottomrule
\end{tabular}
\caption{Frontend Framework Stack}
\end{table}

\subsection{State Management}
\begin{itemize}
    \item React Hooks (useState, useEffect)
    \item Context API for global state
    \item SessionStorage for data persistence
\end{itemize}

\subsection{Styling}
\begin{itemize}
    \item Modern vanilla CSS with Glassmorphism design language
    \item Responsive layout optimized for dark/light mode
    \item CSS custom properties for theming
\end{itemize}

\section{Infrastructure}

\subsection{Database}
\begin{itemize}
    \item \textbf{MongoDB}: NoSQL database for flexible schema management
    \item MongoDB Atlas for cloud deployment
    \item Polymorphic clinical data storage
\end{itemize}

\subsection{Containerization}
\begin{itemize}
    \item \textbf{Docker}: Container runtime
    \item \textbf{Docker Compose}: Multi-container orchestration
    \item Nginx for frontend static file serving
\end{itemize}

\subsection{Dependencies Summary}

\begin{lstlisting}[caption=Key Backend Dependencies]
# Core
flask>=3.0.0
flask-cors>=4.0.0
pymongo>=4.6.0
google-generativeai>=0.3.0

# Data Processing
numpy>=1.21.0
pandas>=1.3.0

# PDF Generation
reportlab>=4.0.0
PyPDF2>=3.0.0

# Security
cryptography>=41.0.0

# Italian Fiscal Code
codicefiscale>=0.9
\end{lstlisting}

\newpage

%==========================================
% CHAPTER 5: CORE COMPONENTS
%==========================================
\chapter{Core Components}

\section{Patient Management}

The module manages the entire patient lifecycle with support for:

\begin{itemize}
    \item \textbf{Italian Citizens}: Identification via validated Fiscal Code
    \item \textbf{Foreign Citizens}: Automatic unique ID generation (format: XX-YYYY)
    \item Automatic age and metadata calculation
    \item Allergy and permanent disease management
\end{itemize}

\section{Doctor Authentication}

Authentication system for doctors with:

\begin{itemize}
    \item Automatically generated unique mnemonic ID (e.g., MR7X9Z)
    \item SHA-256 password hashing
    \item Session management with secure cookies
    \item Specialization and affiliated hospital support
\end{itemize}

\section{Clinical Records}

Clinical records include:

\begin{itemize}
    \item Visit information (ID, timestamp, priority)
    \item Chief complaint and symptoms
    \item Vital signs (blood pressure, heart rate, temperature, saturation)
    \item Attachments (medical images, documents)
    \item Clinical notes
\end{itemize}

\subsection{Vital Signs Reference}

\begin{table}[h]
\centering
\begin{tabular}{lll}
\toprule
\textbf{Parameter} & \textbf{Unit} & \textbf{Normal Range} \\
\midrule
Blood Pressure & mmHg & 90/60 - 120/80 \\
Heart Rate & bpm & 60 - 100 \\
Temperature & °C & 36.1 - 37.2 \\
O2 Saturation & \% & 95 - 100 \\
\bottomrule
\end{tabular}
\caption{Vital Signs Ranges}
\end{table}

\section{PDF Export}

Professional report generation with ReportLab including patient data, clinical records, attached images and AI diagnoses.

\newpage

%==========================================
% CHAPTER 6: AI AND DIAGNOSTICS ENGINE
%==========================================
\chapter{AI and Diagnostics Engine}

\section{Overview}

The AI module provides structured diagnostic support to healthcare professionals through integration with Google Gemini.

\section{Multimodal Capabilities}

The system analyzes simultaneously:

\begin{itemize}
    \item \textbf{Structured Clinical Data}: Patient demographics, medical history, current symptoms, vital signs
    \item \textbf{Medical Images}: X-rays, ECGs, dermatological photographs (Base64 format, automatic RGB conversion)
\end{itemize}

\section{Structured Output}

The AI generates complete clinical assessments including:

\begin{enumerate}
    \item Clinical data analysis and image interpretation
    \item Presumptive diagnosis with probability and clinical reasoning
    \item Differential diagnoses with supporting evidence
    \item Recommended diagnostic tests
    \item Treatment plan (pharmacological and non-pharmacological)
    \item Monitoring and follow-up plan
    \item Urgency assessment and intervention timeline
\end{enumerate}

\section{Error Handling}

Implementation of retry logic with exponential backoff (max 3 attempts) to ensure service reliability.

\section{Medical Chatbot}

Dedicated AI instance for medical queries with:
\begin{itemize}
    \item Isolated API key
    \item Conversational interface
    \item Strict medical-only policy
    \item Real-time response generation
\end{itemize}

\newpage

%==========================================
% CHAPTER 7: SECURITY AND PRIVACY
%==========================================
\chapter{Security and Privacy}

\section{Authentication System}

\subsection{Doctor Registration}
\begin{itemize}
    \item Unique mnemonic ID generation
    \item Password validation (minimum 6 characters)
    \item SHA-256 hashing
    \item Duplicate ID prevention
\end{itemize}

\subsection{Session Management}
Secure cookies with HTTPS-only configuration in production, HttpOnly, SameSite policy and 7-day duration.

\section{Data Protection}

\subsection{Data at Rest}
\begin{itemize}
    \item PII separation from clinical data in MongoDB
    \item Encryption options via MongoDB Atlas
    \item Secure database credentials management
\end{itemize}

\subsection{Data in Transit}
\begin{itemize}
    \item Forced HTTPS in production
    \item TLS/SSL for database connections
    \item Secure cookie transmission
\end{itemize}

\subsection{Anonymization}
Anonymization functionality for external processing: removal of identifying data while maintaining relevant clinical information (allergies, diseases).

\section{Audit Logging}

Logging of all significant actions:
\begin{itemize}
    \item Patient record creation/modification
    \item Clinical record access
    \item Export operations
    \item Authentication events
    \item AI diagnostic requests
\end{itemize}

\section{Compliance}

The system is designed considering:
\begin{itemize}
    \item \textbf{HIPAA}: Health Insurance Portability and Accountability Act
    \item \textbf{GDPR}: General Data Protection Regulation
    \item Audit trail maintenance
    \item Data minimization
\end{itemize}

\newpage

%==========================================
% CHAPTER 8: DATABASE ARCHITECTURE
%==========================================
\chapter{Database Architecture}

\section{MongoDB Collections}

The database uses 5 main collections:

\begin{itemize}
    \item \textbf{doctors}: Doctor credentials and profiles (unique doctor\_id)
    \item \textbf{patients}: Patient demographics (unique codice\_fiscale/patient\_id)
    \item \textbf{patient\_records}: Clinical records (unique encounter\_id, linked to patient\_id)
    \item \textbf{audit\_logs}: Immutable system action history
    \item \textbf{chatbot\_sessions}: Medical chatbot sessions
\end{itemize}

\section{Indexing Strategy}

\begin{table}[h]
\centering
\begin{tabular}{lll}
\toprule
\textbf{Collection} & \textbf{Field} & \textbf{Type} \\
\midrule
patients & codice\_fiscale & Unique \\
patients & patient\_id & Unique \\
patient\_records & encounter\_id & Unique \\
patient\_records & patient\_id & Regular \\
doctors & doctor\_id & Unique \\
audit\_logs & timestamp & Regular \\
\bottomrule
\end{tabular}
\caption{Database Indexes}
\end{table}

\section{Connection Configuration}

MongoDB connection with TLS/SSL, configurable timeouts and connection pooling for optimal performance.

\newpage

%==========================================
% CHAPTER 9: DEPLOYMENT GUIDE
%==========================================
\chapter{Deployment Guide}

\section{Docker Deployment}

\subsection{Quick Start}
\begin{lstlisting}[language=bash]
git clone repository
cp .env.example .env
# Configure environment variables
docker-compose up -d --build
\end{lstlisting}

\section{Manual Deployment}

\subsection{Backend}
\begin{lstlisting}[language=bash]
cd backend
python -m venv venv
venv\Scripts\activate  # Windows
pip install -r requirements.txt
python webapp/app.py
\end{lstlisting}

\subsection{Frontend}
\begin{lstlisting}[language=bash]
cd frontend
npm install
npm run dev
\end{lstlisting}

\section{Environment Variables}

\begin{table}[h]
\centering
\begin{tabular}{lp{8cm}}
\toprule
\textbf{Variable} & \textbf{Description} \\
\midrule
GEMINI\_API\_KEY & Google Gemini API key \\
MONGODB\_CONNECTION\_STRING & MongoDB Atlas connection string \\
FLASK\_SECRET\_KEY & Session encryption key \\
FRONTEND\_URL & Frontend URL (CORS) \\
VITE\_API\_URL & Backend API URL \\
\bottomrule
\end{tabular}
\caption{Required Environment Variables}
\end{table}

\section{Troubleshooting}

\begin{itemize}
    \item \textbf{CORS}: Check protocol and trailing slashes
    \item \textbf{AI}: Images under 4MB, standard formats
    \item \textbf{Database}: IP whitelist MongoDB Atlas
    \item \textbf{Sessions}: Cookie configuration cross-domain
\end{itemize}

\newpage

%==========================================
% CHAPTER 10: FUTURE IMPROVEMENTS
%==========================================
\chapter{Future Improvements}

\section{Enhanced Security}

\begin{itemize}
    \item Advanced password hashing (bcrypt/Argon2)
    \item Multi-Factor Authentication (TOTP, SMS, FIDO2)
    \item JWT-based authentication with token rotation
    \item Granular RBAC (Admin, Doctor, Nurse, Receptionist)
\end{itemize}

\section{Native Applications}

\begin{itemize}
    \item \textbf{Desktop}: Electron, Tauri or Flutter Desktop
    \item \textbf{Mobile}: React Native or Flutter
    \item \textbf{PWA}: Service Workers, offline mode, push notifications
\end{itemize}

\section{OCR Integration}

OCR integration (Tesseract, Google Cloud Vision, Azure) for text extraction from:
\begin{itemize}
    \item Medical prescriptions
    \item Laboratory reports
    \item Discharge letters
    \item Insurance documents
\end{itemize}

\section{Additional Features}

\begin{itemize}
    \item \textbf{HL7 FHIR}: Interoperability with other healthcare systems
    \item \textbf{Analytics Dashboard}: Population statistics, disease trends
    \item \textbf{Telemedicine}: Video consultations, secure messaging
    \item \textbf{Enhanced AI}: Multiple providers, specialty-specific models
    \item \textbf{Appointment Scheduling}: Appointment and resource management
    \item \textbf{Prescription Management}: e-Prescription, drug interaction checking
    \item \textbf{Performance}: Redis caching, CDN, query optimization
    \item \textbf{Accessibility}: WCAG 2.1 AA compliance
\end{itemize}

\section{Implementation Roadmap}

\begin{table}[h]
\centering
\begin{tabular}{llc}
\toprule
\textbf{Phase} & \textbf{Feature} & \textbf{Priority} \\
\midrule
Q1 & Enhanced Security, OCR & High \\
Q2 & PWA, HL7 FHIR & Medium \\
Q3 & Desktop App, Analytics & Medium \\
Q4 & Telemedicine, Mobile & Low \\
\bottomrule
\end{tabular}
\caption{Implementation Roadmap}
\end{table}

\newpage

%==========================================
% CHAPTER 11: CONCLUSION
%==========================================
\chapter{Conclusion}

\section{Summary}

MAV represents an advanced clinical decision support system, combining modern web technologies with artificial intelligence to assist healthcare professionals.

Key achievements:

\begin{itemize}
    \item \textbf{Robust Architecture}: Gateway pattern with well-defined design patterns
    \item \textbf{AI Diagnostics}: Google Gemini integration for multimodal analysis
    \item \textbf{Patient Management}: Complete management with support for Italian and foreign citizens
    \item \textbf{Security-First}: Authentication, authorization and audit logging
    \item \textbf{Modern Deployment}: Docker containerization
\end{itemize}

\section{Technical Excellence}

The project demonstrates adherence to best practices:

\begin{itemize}
    \item Clean code architecture with separation of concerns
    \item Design pattern implementation
    \item Error handling with retry mechanisms
    \item Scalable database design
\end{itemize}

\section{Future Vision}

The roadmap positions MAV for continuous evolution:

\begin{itemize}
    \item Enterprise-grade enhanced security
    \item Native applications for accessibility
    \item OCR for paper documents
    \item Healthcare interoperability standards
\end{itemize}

\vspace{2cm}
\hrule
\vspace{0.5cm}
\begin{center}
\textit{Document generated on \today}
\end{center}

%==========================================
% APPENDICES
%==========================================
\appendix

\chapter{API Reference}

\section{Principali Endpoint}

\begin{table}[h]
\centering
\begin{tabular}{llp{6cm}}
\toprule
\textbf{Method} & \textbf{Endpoint} & \textbf{Description} \\
\midrule
POST & /api/auth/register & Registrazione medico \\
POST & /api/auth/login & Login medico \\
POST & /api/patient/search & Ricerca paziente \\
POST & /api/patient/create & Creazione paziente \\
POST & /api/record/add & Aggiunta cartella clinica \\
POST & /api/diagnosis/generate & Generazione diagnosi AI \\
GET & /api/export/:fiscal\_code & Export PDF \\
\bottomrule
\end{tabular}
\caption{API Endpoints Principali}
\end{table}

\end{document}
